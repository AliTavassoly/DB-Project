\documentclass{article}
\usepackage{graphicx}
\usepackage{caption}
\usepackage{wrapfig}
\usepackage{amsmath}
\usepackage[margin=1in]{geometry}
\usepackage{xcolor}
\usepackage{enumerate}
\usepackage{xepersian}
\usepackage{amssymb}
\usepackage{cancel}
\settextfont[Scale=1]{XB Niloofar.ttf}
\date{}
\title{ سامانه استخدام و کسب مهارت }
\begin{document}
\maketitle
هدف این سامانه، کاریابی، آموزش و  ساخت رزومه است. کلاس ها در این سامانه مانند یک دوره آموزشی میباشند که در رزومه فرد لحاظ میشوند.
\section{حداقل موجودیت های سیستم}
    \begin{itemize}
        \item کاربر عادی: کدکاربری، نام ، نام خانوادگی، ایمیل و رزومه
        \item شرکت ها:نام شرکت، کد ثبت شرکت، آدرس، شماره تماس و اطلاعاتی از این دست
        \item آگهی ها: شرکت، تاریخ شروع، تاریخ انقضا، متن آگهی،سابقه کار مورد نیاز، مهارت های حداقلی، ساعت کاری، تک تکنولوژی مورد نیاز
        \item کلاس : یک کلاس شامل تعدادی تکلیف ،محتوای اموزشی، کاربرهای ارائه دهنده و کاربر های دانشجو ، بازه برگزاری و وضعیت فعلی است
        \item آزمون: یک آزمون شامل تعدادی کاربر برگزار کننده، شرکت کننده، سوال و زمان برگزاری است، نام شرکت برگزار کننده
        \item سوالات: هر سوال شامل تگ ، صورت سوال، سطح سختی،و آزمونی که در آن آمده است.
    \end{itemize}
\section{نیازمندی های سیستم}
    \begin{enumerate}[\hspace{1cm}1.]
        \item کاربران باید بتوانند آگهی ها را با فیلتر هایی روی تگ ها و دیگر اطلاعات آگهی جدا کرده بر و برحسب صفت مورد نظرشان مرتب سازی کرده و در چند صفحه ببینند
        
        \item شرکت می تواند بعد از آزمون، به سوال ها با تگ مشخص شده ضریب داده، و در کنار اطلاعات و سابقه هر درخواست دهنده قرار دهد تا امتیاز نهایی برای آنها محاسبه کرده و به تعداد دلخواه کاربر های برتر را جدا کند.
        \item هر شرکت ممکن است درخواستی را رد کند یا وارد مرحله ازمون یا مصاحبه کند.
        
        \item آگهی
        \begin{itemize}
            \item شرکت ها باید بتوانند آگهی روی سامانه قرار دهند 
            \item سابقه کار میتواند شامل \lr{junior}، \lr{senior} یا ... باشد.
            \item ساعت کاری میتواند پاره وقت، تمام وقت یا ... باشد. در صورت پاره وقت بودن، تایم مورد نظر باید نوشته شود.
            \item مهارت های حداقلی سطح مورد نیاز برای هر تکنولوژی را بیان میکند.
        \end{itemize}

        \item قابلیت نوتیفیکیشن: برای شرکت و کاربر در هر مرحله از درخواست یک نوتیفیکشن فرستاده می شود
        \begin{itemize}
            \item برای کاربر پس از تایید یا رد شدن توسط شرکت یا تعیین شدن زمان آزمون یا مصاحبه نوتیفیکیشن فرستاده می شود
            \item برای شرکت نیز برای درخواست های ارسالی نوتیفیکیشن قرار داده می شود
            \item قابلیت اطلاع رسانی آگهی های جدید:کاربر می توانند با قرار دادن فیلتر روی تگ و دیگر صفات آگهی ها
            از آگهی های جدید مطابق با فیلتر قرار داده شده با خبر شود.
        \end{itemize}

        \pagebreak

        \item رزومه:هر کاربر قبل از شناخته شدن به عنوان کاربر سیستم باید رزومه خودرا وارد کند اما این رزومه قابل تدوین مجدد است
        \item محتوای رزومه
        \begin{itemize}
            \item اطلاعات فردی:نام، نام خانوادگی، ایمیل و شماره تماس، استان وشهر، تاریخ تولد، جنسیت، وضعیت تاهل، متن "درباره من"
            \item اطلاعات شغلی:وضعیت اشتغال، سابقه کار، شغل مورد نظر، پاره وقت یا تمام وقت(در صورت پاره وقت بودن ساعت کاری به ازا هر هفته)، حقوق مورد انتظار،تکنولوژی های مورد علاقه
            \item سوابق شغلی و تحصیلی: هر سابقه شامل نام موسسه،درس، نمره کسب شده، لینک به صفحه ای که برای تایید نمره می توان به آن رجوع کرد(در صورت وجود)
            \item مهارت ها: زبان ها (شامل زبان، سطح توانایی در ۴مهارت آن زبان) مهارت های فنی:یک مهارت فنی شامل تگ آن مهارت، و سطح توانایی است
            \item پروژه ها: لینک به محل قرار گیری پروژه های پیاده سازی شده
            \item سامانه روی رزومه کاربر سوابق توانایی های فنی وی که از استخدامی های قبلی و آزمون هایی که در آن شرکت کرده بدست می آید را قرار دهد. و کاربر توانایی حذف این سوابق از رزومه را ندارد     
            \item همچنین سطح و تعداد سوالاتی که کاربر حل کرده در رزومه او قید می‌شود
        \end{itemize}
        
        \item سوالات و آزمون ها
        \begin{itemize}
            \item  یک آزمون شامل تعدادی سوال است، برای هر سوال یک ضریب و یک تگ تکنولوژی وجود دارد
            \item آزمون ها می توانند خصوصی یا عمومی باشند، آزمون های خصوصی کاربر های مشخصی را می پذیرند
            \item شرکت برگزار کننده آزمون میتواند مقدار \lr{NULL} بگیرد
            
            \item هر سوال مربوط به یک آزمون است
            \item سوالاتی که متعلق به آزمون های عمومی هستند در دسترس عموم هستند و سوالات مربوط به آزمون های خصوصی را فقط شرکت کنندگان آزمون میتوانند مشاهده کنند
        \end{itemize}
        
        \item کلاس ها
        \begin{itemize}
            \item کاربر های ارائه دهنده کلاس ها می توانند از انواع استاد، دستیار(با قابلیت های محدود به یک مصحح ، طراح، تولید کننده محتوا یا ترکیبی از اینها) و دانشجو باشد.
            \item داخل کلاس می توان آزمون خصوصی، تکلیف یا محتوا های آموزشی قرار داد
            \item تکلیف ها بازه زمانی(با قابلیت تحویل با تاخیر در صورت وجود)، محدودیت حجم و محدودیت فرمت فایل دارند
            \item هر کلاس تاریخ برگزاری و بارم بندی دارد که این بارم بندی به هر تکلیف و آزمون بخشی از نمره را اختصاص می دهد
            \item در انتهای برگزاری کلاس به هر دانشجو نمره داده شده و سطح توانمندی ارزیابی شده در کلاس به رزومه اضافه می شود.
        \end{itemize}
        
    \end{enumerate}
\end{document}