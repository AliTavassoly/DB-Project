\documentclass{article}

\usepackage{graphicx}
\usepackage{listings}
\usepackage{amsmath}
\usepackage{xcolor}
\usepackage{enumerate}
\usepackage{hyperref}
\usepackage{xepersian}
\usepackage{amssymb}    
\usepackage[dash,dot]{dashundergaps}
\settextfont{XB Niloofar.ttf}


\title{توضیحات پروژه}
\date{}
\begin{document}
\maketitle

\section{فاز اول}
    \begin{itemize}
        \item    در فاز اول پروژه، اقدام به طراحی \lr{ERD} کردیم.
    \end{itemize}
\section{فاز دوم}
    \begin{itemize}
    \item در فاز دوم پروژه، طراحی منطقی را انجام دادیم. یعنی ابتدا برای موجودیت‌ها جداول متناظرشان را تعریف کردیم، سپس روابط برای روابط بین موجودیت‌ها نیز جداولی ساختیم.
        \\
    \item در این فاز سعی شد جداول طراحی شده به گونه‌ای باشند که تا حد 
        \lr{3NF} نرمال باشند.
        \\
    \item از طرفی، جداول به ترتیبی در فایل نوشته شده‌اند که اگر یک جدول دارای کلید خارجی باشد، جدولی که آن کلید خارجی به عنوان کلید اصلی آن است، در بالای جدول فوق تعریف شده باشد.
        بنابراین از نظر ارجاع دادن، در گراف ارجاع دور نداریم.
    \end{itemize}
\section{فاز سوم}
    \begin{itemize}
        \item در این فاز، جداول طراحی شده در فاز دوم را در سیستم مدیریت پایگاه داده \lr{MySQL} پیاده کردیم.
        \item در این فاز، به این توجه کردیم که اگر یک موجودیت حذف یا بروزرسانی شود، چه اتفاقی برای جداولی می‌افتد که به آن ارجاع میدهند و از انواع \lr{RESTRICTED} و یا \lr{CASCADE} و ... استفاده کردیم.
    \end{itemize}

    \subsection{توضیحاتی در مورد \lr{TRIGGER}ها}
        \begin{itemize}
            \item اولین تریگر به اینصورت است که اگر یک آگهی جدید به سیستم اضافه شد، به همه‌ی کاربرهایی که به این آگهی احتمالا علاقه مند هستند یک نوتیفیکیشن ارسال میشود.
            \item دومین تریگر به اینصورت است که اگر وضعیت یک درخواست تغییر کرد، یعنی اگر یک کاربر به شرکتی درخواست فرستاده بود و وضعیت آن قبول و یا رد شد، این تغییر وضعیت به کاربر ارسال میشود.
            \item سومین تریگر به اینصورت است که اگر کاربری به یک شرکت درخواستی فرستاد، به آن شرکت یک نوتیفیکیشن ارسال میشود.
            \item چهارمین تریگر به اینصورت است که اگر کاربری سوالی را حل کرد، آن سوال به رزومه‌اش ارسال می‌شود.
            \item پنجمین تریگر به اینصورت است که اگر کاربر در یک کانتست شرکت کرد، در رزومه‌اش ثبت میشود(به همراه نمره گرفته شده و رتبه کاربر در کانتست)
            \item ششمین تریگر به اینصورت است که اگر رتبه و نمره کاربر در کانتست عوض شد، در رزومه هم این تغییرات اعمال شود. برای مثال اگر در حین کانتست، کاربر سوالی را اپسکت کند، نمره و رتبه کاربر تغییر میکند. این تغییرات به صورت آنلاین در رزومه هم اعمال میشود.
        \end{itemize}
    \subsection{توضیحاتی در مورد \lr{VIEW}ها}
        ما در قسمت دیدها، برای هر کاربر که به تازگی وارد سیستم میشود، یک سری دید مختص آن کاربر میسازیم. اینکار را به کمک \lr{PROCEDURE}ها انجام میدهیم.
        روند کار به این صورت است که یک سری \lr{PROCEDURE} داریم که اگر یک کاربر جدید ساخته شد، آن را صدا میزنیم. \lr{PROCEDURE} مورد نظر، ابتدا کاربر با مشخصات دریافتی را در جدول \lr{USER} اضافه میکند، سپس دیدهایی که کاربر لازم دارد را برای آن میسازد.
        در این دیدها، آیدی کاربر آمده تا بتوان به وسیله آن، نام‌های متمایز را برای دیدها انتخاب کرد. در موارد زیر، به اختصار دیدهایی که ساخته‌ایم را توضیح میدهیم.
        البته دیدهای زیادی میتوانستیم بسازیم، اما به علت تعداد زیاد آنها، ما به موارد زیر بسنده کردیم و ساختن بقیه دیدها با توجه به \lr{PROCEDURE}های ساخته شده، کار سختی نخواهد بود.
        همین کار را باید برای شرکت‌ها هم انجام بدهیم که قالب آن دقیقا مانند قالبی است که برای \lr{USER}ها ساخته‌ایم.
        \begin{itemize}
            \item
            \item
        \end{itemize}

\section{تمرکز افراد گروه}
تقریبا در همه موارد همه اعضای گروه حضور داشتند و هر سه عضو گروه تسلط کافی به تمام قسمت‌های پروژه دارند.

\end{document}


